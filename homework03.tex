\documentclass{article}
\usepackage{amsmath,amsthm,latexsym}
\usepackage[margin=1in]{geometry}

\theoremstyle{definition}
\newtheorem{problem}{Problem}
\newtheorem*{solution}{Solution}
\newtheorem*{resources}{Resources}

\newcommand{\name}[1]{\noindent\textbf{Name: #1}}
\newcommand{\honor}{\noindent On my honor, as an Aggie, I have neither given nor received any unauthorized aid on any portion of the academic work included in this assignment. Furthermore, I have disclosed all resources (people, books, web sites, etc.) that have been used to prepare this homework. 
\\[2ex]
\textbf{Signature:} \underline{\hspace*{7cm}} }

 
\newcommand{\checklist}{\noindent\textbf{Checklist:}
\begin{enumerate}
\item[$(~)$] Did you add your \textbf{name}?
\item[$(~)$] Did you disclose all \textbf{resources} that you have used? \\
(This includes all people, books, websites, etc.\ that you have consulted)
\item[$(~)$] Did you \textbf{sign} that you followed the Aggie honor code? 
\item[$(~)$] Did you solve \textbf{every problem}? 
\item[$(~)$] Did you submit the PDF file of your homework on \textbf{eCampus}?
\item[$(~)$] Did you submit a \textbf{signed and stapled} hardcopy of the PDF file \textbf{in class}? 
\end{enumerate}
}

\newcommand{\problemset}[1]{\begin{center}\textbf{Problem Set #1}\end{center}}
\newcommand{\duedate}[2]{\begin{quote}\textbf{Due dates:} Electronic
    submission of the PDF file for this homework is due on
    \textbf{#1} on \texttt{http://ecampus.tamu.edu}.  A signed paper copy of the PDF is due on
    \textbf{#2} at the beginning of class.\end{quote} }


\begin{document}
\begin{center}
{\large
CSCE 222 [503] Discrete Structures for Computing\\[.5ex]
Spring 2015 -- Philip C. Ritchey\\}
\end{center}

\problemset{3}

\duedate{2/12/2015 (Thursday) before 11:59 p.m.}{2/13/2015 (Friday)}

\name{ (Han Hong) }

\begin{resources} Discrete Mathematics and Its Applications by Rosen, (additional people, books, articles, web pages, etc.\ that
  have been consulted when producing this homework)
\end{resources}

\honor

\bigskip

\begin{problem} (10 points)
Section 2.1, Exercise 4, page 125
\end{problem}
\begin{solution}~\\
a.\ Second is a subset of first. \\
b.\ Second is a subset of first. \\
c.\ Neither is a subset of the other. \\
\end{solution}

\begin{problem} (10 points)
Section 2.1, Exercise 10, page 125
\end{problem}
\begin{solution}~\\
a.\ F \\
b.\ T \\ 
c.\ T \\
d.\ T \\
e.\ F \\
f.\ F \\
g.\ T \\
\end{solution}

\begin{problem} (10 points)
Section 2.1, Exercise 34, page 126
\end{problem}
\begin{solution}~\\
a.\ {(a,a,a)} \\
b.\ {(0,0,0),(0,0,a),(0,a,0),(a,0,0),(a,a,0),(a,0,a),(0,a,a),(a,a,a)} \\
\end{solution}

\begin{problem} (10 points)
Section 2.2, Exercise 2, page 136
\end{problem}
\begin{solution}~\\
a.\ $A \cap B$ \\
b.\ $A - B$ \\
c.\ $A \cup B$ \\
d.\ $\overline A \cup \overline B = \overline(A \cap B)$ \\
\end{solution}

\begin{problem} (10 points)
Section 2.2, Exercise 8, page 136
\end{problem}
\begin{solution}~\\
a.\ $A \cup B = {x|(x \in A) \lor (x \in A)}$ This is just a set A itself \\
b.\ $A \cap B = {x|(x \in A) \land (x \in A)}$ This is just a set A itself \\
\end{solution}

\begin{problem} (10 points)
Section 2.2, Exercise 16, page 136
\end{problem}
\begin{solution}~\\
a.\ Assume $x \in (A ∩ B)$ so $x \in A and x \in B$. So, $x \in A$. Since $x \in (A ∩ B)$ thus $x \in A$ and $(A \cup B) \subseteq A$. \\
b.\ Assume $x \in A$. Then certainly $(x \in A) \lor (x \in B)$. So, $x \in A \cup B$. And since \\
   $x \in A \to x \in A \cup B$, then $A \subseteq A \lor B$
c.\ Assume $x \in A-B$ then $(x \in A)\land(x \notin B)$ making $x \in A$ \\
    thus $(x \in (A-B)) \to x \in A$, so $A-B \subseteq A$ \\
d.\ Assume $x \in A \cap (B - A)$. So $x \in A and x \in B-A$. Because of $x \in B-A then x \in B$ \\
   and $x \notin A$. But $x \in A!$ Contradiction, so our assumption $x \in A \cap (B-A)$ must be \\
   wrong. Thus $A \cap (B - A)$ must be empty.
e.\ Assume $x \in A \lor (B-A)$. Then $x \in A or x \in B \ A$. 
   If $x \in B \ A,(x \in B) \lor (x 6\in A)$, so $x \in B$. So $x \in A or x \in B$, \\
   which implies $x \in A \lor B$. Thus, $x \in A \lor (B - A) \to x \in A \lor B$, \\
   or $A \lor (B - A) ⊆ A \lor B $ \\
\end{solution}

\begin{problem} (10 points)
Section 2.2, Exercise 46, page 136
\end{problem}
\begin{solution}~\\

\end{solution}

\begin{problem} (10 points)
Section 2.3, Exercise 22, page 153
\end{problem}
\begin{solution}~\\
a.\ Bijection \\
b.\ No \\
c.\ No \\
d.\ Bijection \\
\end{solution}

\begin{problem} (10 points)
Section 2.3, Exercise 42, page 154
\end{problem}
\begin{solution}~\\
a. {1, −1} \\
b. {x| − 1 < x < 0 $\lor$ 0 < x < 1} \\
c. {x|x > 2 $\lor$ x < −2} \\
\end{solution}

\begin{problem} (10 points)
Section 2.3, Exercise 56, page 154
\end{problem}
\begin{solution}
\item  $\floor(b)-\floor(a)$
\item  $\ceiling(b)-\floor(a)+1$
\item  $\ceiling(b)-\floor(a)$
\item  $\floor(b-a)$
\item  None of above. 
\end{solution}

\goodbreak
\checklist
\end{document}