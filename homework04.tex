\documentclass{article}
\usepackage{amsmath,amsthm,latexsym}
\usepackage[margin=1in]{geometry}

\theoremstyle{definition}
\newtheorem{problem}{Problem}
\newtheorem*{solution}{Solution}
\newtheorem*{resources}{Resources}

\newcommand{\name}[1]{\noindent\textbf{Name: #1}}
\newcommand{\honor}{\noindent On my honor, as an Aggie, I have neither given nor received any unauthorized aid on any portion of the academic work included in this assignment. Furthermore, I have disclosed all resources (people, books, web sites, etc.) that have been used to prepare this homework. 
\\[2ex]
\textbf{Signature:} \underline{\hspace*{7cm}} }

 
\newcommand{\checklist}{\noindent\textbf{Checklist:}
\begin{enumerate}
\item[$(~)$] Did you add your \textbf{name}?
\item[$(~)$] Did you disclose all \textbf{resources} that you have used? \\
(This includes all people, books, websites, etc.\ that you have consulted)
\item[$(~)$] Did you \textbf{sign} that you followed the Aggie honor code? 
\item[$(~)$] Did you solve \textbf{every problem}? 
\item[$(~)$] Did you submit the PDF file of your homework on \textbf{eCampus}?
\item[$(~)$] Did you submit a \textbf{signed and stapled} hardcopy of the PDF file \textbf{in class}? 
\end{enumerate}
}

\newcommand{\problemset}[1]{\begin{center}\textbf{Problem Set #1}\end{center}}
\newcommand{\duedate}[2]{\begin{quote}\textbf{Due dates:} Electronic
    submission of the PDF file for this homework is due on
    \textbf{#1} on \texttt{http://ecampus.tamu.edu}.  A signed and stapled paper copy of the PDF is due on
    \textbf{#2} at the beginning of class.\\You must show your work.  \textbf{No work $\to$ no credit.}\end{quote} }


\begin{document}
\begin{center}
{\large
CSCE 222 [503] Discrete Structures for Computing\\[.5ex]
Spring 2015 -- Philip C. Ritchey\\}
\end{center}

\problemset{4}

\duedate{2/19/2015 (Thursday) before 11:59 p.m.}{2/20/2015 (Friday)}

\name{ (Han Hong) }

\begin{resources} Discrete Mathematics and Its Applications by Rosen, (additional people, books, articles, web pages, etc.\ that
  have been consulted when producing this homework)
\end{resources}

\honor

\bigskip

\begin{problem} (10 points)
Section 9.1, Exercise 4, page 581
\end{problem}
\begin{solution}~\\
a) a is taller than b \\
\indent	Reflexive: no (a can't be taller than a) \\
\indent	Symmetric: no (a is taller than b, b cannot be taller than a) \\
\indent	Antisymmetric: yes (No condition to disprove) \\
\indent Transitive: yes (if a is taller than b and b is taller than c, means a is taller than c) \\
b) a and b were born on the same day \\
\indent Reflexive: yes (a = a, b = b) \\
\indent Symmetric: yes (if a = b, then b must = a) \\
\indent Antisymmetric: no (a and c can be on the same day) \\
\indent Transitive: yes (if a = b, and b = c, then a = c)\\
c) a has same first name as b\\
\indent Reflexive: yes (a = a, b = b)\\
\indent Symmetric: yes (if a = b, then b = a)\\
\indent Antisymmetric: no (a != b, and still belong to the set)\\
\indent Transitive: yes (if a = b, b = c, then a = c)
d) a and b have a common grandparent\\
\indent Reflexive: yes (a = a, b = b)\\
\indent Symmetric: yes (if a = b, then b = a)
\indent Antisymmetric: no (a != b, and still belong to the set)\\
\indent Transitive: no (a's grandparent doesn't need to be the same for b and c grandparent)\\

\end{solution}

\begin{problem} (10 points)
Section 9.1, Exercise 26, page 582
\end{problem}
\begin{solution}~\\
a) {(a, b) | b > a} \\
b) {(a, b) | a <= b} \\
\end{solution}

\begin{problem} (10 points)
Section 9.1, Exercise 34, page 582
\end{problem}
\begin{solution}~\\
a) $R^{2}$ \indent b) $R_{5}$ \\
c) $R_{2}$ \indent d) $R_{4}$ \\
e) $\phi$  \indent f) $R_{5}$ \\ 
g) $R_{5}$ \indent h) $R_{6}$ \\
\end{solution}

\begin{problem} (10 points)
Section 9.1, Exercise 36, page 582
\end{problem}
\begin{solution}~\\
a) $R_{1}$ \indent b) $R_{1}$ \\
c) $R^{2}$ \indent d) $R^{2}$ \\
e) $R_{1}$ \indent f) $R^{2}$ \\
g) $R^{2}$ \indent h) $R_{3}$ \\
\end{solution}



\begin{problem} (10 points)
Section 9.2, Exercise 2, page 589
\end{problem}
\begin{solution} ~\\
(6, 1, 1, 1) \\
(1, 6, 1, 1) \\
(1, 1, 6, 1) \\
(1, 1, 1, 6) \\
(2, 3, 1, 1) \\
(3, 2, 1, 1) \\
(2, 1, 3, 1) \\
(3, 1, 2, 1) \\
(2, 1, 1, 3) \\
(3, 1, 1, 2) \\
(1, 2, 3, 1) \\
(1, 3, 2, 1) \\
(1, 2, 1, 3) \\
(1, 3, 1, 2) \\
(1, 1, 2, 3) \\
(1, 1, 3, 2) \\
\end{solution}



\begin{problem} (10 points)
Section 9.3, Exercise 32, page 597.\\(Definitions of \textit{irreflexive} and \textit{asymmetric} are on pages 581--582)
\end{problem}
\begin{solution}~\\
Graph 26\\
\indent Reflexive: yes \\
\indent Symmetric: no \\
\indent Anti-Symmetric: no \\
\indent Transitive: no \\
\indent Asymmetric: no \\

Graph 27\\
\indent Reflexive: yes \\
\indent Symmetric: yes \\
\indent Anti-Symmetric: no \\ 
\indent Transitive: yes \\
\indent Asymmetric: no \\

Graph 28\\
\indent Reflexive: yes \\
\indent Symmetric: yes \\
\indent Anti-Symmetric: no \\ 
\indent Transitive: yes \\
\indent Asymmetric: no \\
\end{solution}



\begin{problem} (10 points)
Section 9.5, Exercise 2, page 615
\end{problem}
\begin{solution}~\\
Relations a and b are equivalence, with b and c have a same relation with parent.\\
\end{solution}

\begin{problem} (10 points)
Section 9.5, Exercise 18, page 615
\end{problem}
\begin{solution}~\\

\end{solution}


\begin{problem} (10 points)
Section 9.6, Exercise 4, page 630
\end{problem}
\begin{solution}~\\
a) no\\
b) no\\
c) yes\\
d) yes\\
\end{solution}

\begin{problem} (10 points)
Section 9.6, Exercise 16 a) and b), page 630
\end{problem}
\begin{solution}~\\
a) (1,1),(1,2),(1,3),(1,4),(2,1),(2,2)\\
b) (3,2),(3,3),(3,4),(4,1),(4,2),(4,3),(4,4)\\
\end{solution}

\goodbreak
\checklist
\end{document}
