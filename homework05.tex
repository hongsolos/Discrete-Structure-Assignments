\documentclass{article}
\usepackage{amsmath,amsthm,latexsym}
\usepackage[margin=1in]{geometry}

\theoremstyle{definition}
\newtheorem{problem}{Problem}
\newtheorem*{solution}{Solution}
\newtheorem*{resources}{Resources}

\newcommand{\name}[1]{\noindent\textbf{Name: #1}}
\newcommand{\honor}{\noindent On my honor, as an Aggie, I have neither given nor received any unauthorized aid on any portion of the academic work included in this assignment. Furthermore, I have disclosed all resources (people, books, web sites, etc.) that have been used to prepare this homework. 
\\[2ex]
\textbf{Signature:} \underline{\hspace*{7cm}} }

 
\newcommand{\checklist}{\noindent\textbf{Checklist:}
\begin{enumerate}
\item[$(~)$] Did you add your \textbf{name}?
\item[$(~)$] Did you disclose all \textbf{resources} that you have used? \\
(This includes all people, books, websites, etc.\ that you have consulted)
\item[$(~)$] Did you \textbf{sign} that you followed the Aggie honor code? 
\item[$(~)$] Did you solve \textbf{every problem}? 
\item[$(~)$] Did you submit the PDF file of your homework on \textbf{eCampus}?
\item[$(~)$] Did you submit a \textbf{signed and stapled} hardcopy of the PDF file \textbf{in class}? 
\end{enumerate}
}

\newcommand{\problemset}[1]{\begin{center}\textbf{Problem Set #1}\end{center}}
\newcommand{\duedate}[2]{\begin{quote}\textbf{Due dates:} Electronic
    submission of the PDF file for this homework is due on
    \textbf{#1} on \texttt{http://ecampus.tamu.edu}.  A signed and stapled paper copy of the PDF is due on
    \textbf{#2} at the beginning of class.\\You must show your work.  \textbf{No work $\to$ no credit.}\end{quote} }


\begin{document}
\begin{center}
{\large
CSCE 222 [503] Discrete Structures for Computing\\[.5ex]
Spring 2015 -- Philip C. Ritchey\\}
\end{center}

\problemset{5}

\duedate{2/26/2015 (Thursday) before 11:59 p.m.}{2/27/2015 (Friday)}

\name{ (Han Hong) }

\begin{resources} Discrete Mathematics and Its Applications by Rosen, (additional people, books, articles, web pages, etc.\ that
  have been consulted when producing this homework)
\end{resources}

\honor

\bigskip

\begin{problem} (11 points)
Using a proof similar to the one in class for the geometric series, prove that
$$\displaystyle \sum_{i=0}^{n-1} i = \frac{n(n-1)}{2}$$
\end{problem}
\begin{solution}~\\
$$\displaystyle Let S = \sum_{i=0}^{n-1} i$$ \\
$$\displaystyle => 2S = \sum_{i=0}^{n} n+1$$ \\
$$\displaystyle => 2S = n(n+1)$$ \\
$$\displaystyle S = \frac{n(n-1)}{2}$$ \\
\end{solution}

\begin{problem} (11 points)
Derive a general closed-form solution for the sum of an arithmetic progression $\{a_n\}$:
$$\displaystyle \sum_{i=m}^{n} a_i$$
\end{problem}
\begin{solution}~\\
$$\displaystyle \sum_{i=m}^{n} a_i = \sum_{i=0}^{n} a_i - \sum_{i=0}^{m-1} a_i$$ \\
\indent $ =====> \frac{n(n+1)}{2}$ - $\frac{m(m-1)}{2}$
\end{solution}

\begin{problem} (12 points)
Section 2.4, Exercise 17 d, f, h page 168\\
\textit{You must show your \textbf{work} to get \textbf{credit}.}
\end{problem}
\begin{solution}~\\
d. $a_n = a_{n-1} + 2n+3$; a_o = 4\\
\indent   a_o = 4\\
\indent   a_1 = 4 + 2(1) + 3 = 9\\
\indent   a_2 = 9 + 2(2)+3 = 16\\
\indent   => $a_n = (2+n)^2$\\
f. $a_n = 3a_{n-1} + 1$; a_o = 4\\
\indent   a_o = 1\\
\indent  a_1 = 3(1) + 1 = 4\\
\indent   a_2 = 3(4) + 1 = 13\\
\indent   a_3 = 3(13) + 1 = 40\\
\indent   => Geometric series: \frac{3n(n+1)}{2} + 1
h. $a_n = 2na_{n-1}$; a_o = 1\\
\indent   a_1 = 2(1)(1) = 2\\
\indent   a_2 = 2(2)2 = 8 \\
\indent   a_3 = 2(3)(8) = 48 \\
\indent  => $2^nn!$
\end{solution}

\begin{problem} (11 points)
Section 2.4, Exercise 24, page 169
\end{problem}
\begin{solution}~\\
a. B(k) = $\frac {r}{12}$B(k-1) - P\\
b. P = $\frac {r}{12}$B(T-1)\\
\end{solution}

\begin{problem} (11 points)
Section 2.4, Exercise 34, page 169
\end{problem}
\begin{solution}~\\
a. $ => \sum_{i=1}^{3}((i-1)+(i-2))$
\indent = ((1 − 1) + (1 − 2)) + ((2 − 1) + (2 − 2)) + ((3 − 1) + (3 − 2))\\
\indent = 0 − 1 + 1 + 0 + 2 + 1 = 3.\\
b. $ => \sum_{i=0}^{3}$((3i+2*0)+(3i+2*1)+(3i+2*2))\\
\indent = $\sum_{i=0}(9i+6)$\\
\indent =9 * 0 + 6 + 9 * 1 + 6 + 9 * 2 + 6 + 9 * 3 + 6\\
\indent =81\\
c. $ => \sum_{i=1}^{3}$(0+1+2)\\
\indent = $\sum_{i=1}^{3}$3\\
\indent = 3 + 3 + 3 = 9\\

\end{solution}

\begin{problem} (11 points)
Section 2.4, Exercise 36, page 169
\end{problem}
\begin{solution}~\\
$\sum_{k-1}^{n} \frac{1}{k(k+1)}$ = $\sum_{k-1}^{n}$ $\frac {1}{k}$ - $\frac {1}{k+1} = ($\frac {1}{2}$ - $\frac {1}{2}$) + ($\frac {1}{2}$ - $\frac {1}{3}$)... ($\frac {1}{n}$ - $\frac {1}{n+1}$) \\
ALL cancels out, leaves \frac {1}{n+1}
\end{solution}

\begin{problem} (11 points)
Section 2.4, Exercise 40, page 169
\end{problem}
\begin{solution}~\\
Take sum from 1 to 200 minus sum from 1 too 99, will give result from 99 to 200\\
Answer: 379507500\\
\end{solution}

\begin{problem} (11 points)
Section 2.4, Exercise 44, page 170
\end{problem}
\begin{solution}~\\
$\prod\limits_{i=n}^1$i\\
\end{solution}

\begin{problem} (11 points)
Section 2.4, Exercise 46, page 170
\end{problem}
\begin{solution}~\\
$\prod\limits_{n=0}^4$j! = 0! * 1! * 2! * 3! * 4! = 1 * 1 * 2 * 6 * 24 = 288\\
\end{solution}


\goodbreak
\checklist
\end{document}
